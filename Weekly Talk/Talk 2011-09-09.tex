\documentclass{article}
\usepackage{booktabs}
\renewcommand{\baselinestretch}{1.1}
\usepackage{amsthm,amsmath, amssymb, amsfonts, amscd, xspace, pifont, color}
\usepackage{boxedminipage}
\usepackage{epsfig, psfrag}
\usepackage{multirow}
\usepackage[top=1in, bottom=1in, left=1in, right=1in]{geometry}
\usepackage{rotating}
\usepackage[round]{natbib}

\newtheorem{Theorem}{Theorem}[section]
\newtheorem{Lemma}[Theorem]{Lemma}

\newcommand{\FF}{\mathcal{F}}
\newcommand{\GG}{\mathcal{G}}
\newcommand{\LL}{\mathcal{L}}
\newcommand{\RR}{\mathbb{R}}
\newcommand{\Y}{\mathbf{Y}}
\newcommand{\X}{\mathbf{X}}
\newcommand{\Q}{\mathbf{Q}}
\newcommand{\HH}{\mathbf{H}}
\newcommand{\V}{\mathbf{V}}
\newcommand{\VV}{\mathbf{V}^{-1}}
\newcommand{\argmax}{\mathop{\arg\max}}
\newcommand{\argmin}{\mathop{\arg\min}}


\begin{document}

    \begin{center}
        \Large{Talk on 2011-09-09}
    \end{center}

    \fontsize{11pt}{\baselineskip}\selectfont

    \section{Warfarin Data}

    \subsection{Data description}

        For this data set, the sample size is 304. Two phenotypes are measured, labeled as ydose and yinr, respectively. Also, some covariant are recorded: sex, age, height and weight. The genotypes for two genes are provided. Gene A contains 7 SNPs and gene B contains a 4-allelic marker.

    \subsection{Data Analysis}

        Two Trait-Gene Similarity regression models with 3 tests are performed for this data set. The first model contains the Similarity between 2 genes and the similarity for the interaction, as described below:

        \begin{equation*}
            Z_{ij}=b^AS_{ij}^A+b^BS_{ij}^B+b^{AB}S_{ij}^{AB}+e_{ij}
        \end{equation*}
        where $Z_{ij}$ is the normalized trait similarity between individual $i$ and $j$, $S^A_{ij}$ and $S^B_{ij}$ record the gene similarity between individual $i$ and $j$ for gene A and gene B respectively. $S^{AB}_{ij}$ describes the genetic similarity of the interaction between gene A and gene B. $b^A$, $b^B$ and $b^{AB}$ are the effects of $S^A_{ij}$, $S^B_{ij}$ and $S^{AB}_{ij}$, respectively.\\
        \begin{boxedminipage}{\textwidth}
            Q: what is "normalized trait similarity"?
            
            A: $Z_{ij}=(Y_i-\mu_i)^T(Y_j-\mu_j)$ where $\mu_i=E(Y_i|X_i,G_i)=X_i\gamma$
        \end{boxedminipage}

        Since the data is a little different from the usual 2-allelic marker data. The genetic similarity matrix for a gene is computed in the following way:

        Define $M$ is number of the SNPs the gene contains. For $ith$ individual's $mth$ marker, two SNPs are recorded, denoted as $G_{i1}^m$ and $G_{i2}^m$, then we have the genetic similarity $s_{ij}^m$ between individual $i$ and $j$ for the $mth$ marker in a particular gene:

        \begin{equation*}
            s_{ij}^m=\begin{cases}
                1,\quad G_{i1}^m=G_{i2}^m=G_{j1}^m=G_{j2}^m \\
                0,\quad G_{i1}^m\neq G_{j1}^m \mbox{ and } G_{i1}^m\neq G_{j2}^m \mbox{ and } G_{i2}^m\neq G_{j1}^m \mbox{ and } G_{i2}^m\neq G_{j2}^m \\
                0.5,\quad o.w
         \end{cases}
        \end{equation*}

        and the similarity of the gene $S_{ij}$ is
        \[
            S_{ij}=\frac{1}{M}\sum^M_{m=1}s^m_{ij}
        \]

        and $S^{AB}_{ij}=S^A_{ij}\times S^B_{ij}$.\\        
        \begin{boxedminipage}{\textwidth}
            Q: I'm puzzled with why it will become more complicated than biallelic case....
            
            I assume that you are using UNWEIGHTED maxIBS. Then the score is simple, if the $G_i$ = $G_j$, then score "1", if $G_i$ and $G_j$ only have one allele the same, the score "0.5", otherwise "0". The rule is the same as before.... (If you put weight by allele frequency, that will be more complicated...)

            A: No, actually I am using Unweighted "Typical IBS". I think the reason the 4-allelic case seems to be more complicated than biallelic is that for biallelic there are only 3 possibilities while for 4-allelic case there are 10 possibilities which make "only have one allele the same" hard to define.
        \end{boxedminipage}
        
        
        There are two tests under this model. One the joint test under the hypothesis: $b^A=b^B=b^{AB}=0$, the other one is the interaction test under the hypothesis $b^{AB}=0$.

        Other model for this data set is shown below:

        \begin{equation*}
            Z_{ij}=b^AS_{ij}^A+b^BS_{ij}^B+e_{ij}
        \end{equation*}

        In this model, we just consider the main effect of the gene similarity. The goal of this model is to test $b^A=0$ with the adjustment of Gene B.

        \begin{table}[htbp]
            \centering
            \caption{Our method result}
            \begin{tabular}{c|cccc}
                \toprule
                 & & & \multicolumn{2}{c}{One Gene Test}\\
                 \cmidrule{4-5}
                 Traits & Joint test & Epi test & Gene A & Gene B\\
                 \hline
                 ydose & $3.15\times10^{-21}$   &   1   &   $2.6\times 10^{-22}$    &   $3.38\times10^{-8}$\\
                 yinr  & 0.41                  &   0.54&   0.46                    &   0.59            \\
                \bottomrule
            \end{tabular}
        \end{table}
        
        
        \begin{table}[htbp]
            \centering
            \caption{PCA and PLS result}
            \begin{tabular}{c|cccc}
                \toprule
                 & \multicolumn{2}{c}{PCA}& \multicolumn{2}{c}{PLS}\\
                 \cmidrule{2-5}
                 Traits & Joint test & Epi test & Joint test & Epi test\\
                 \hline
                 ydose & $2.2\times10^{-16}$   &   0.6495   &   $2.2\times 10^{-16}$    &   0.312\\
                 yinr  & 0.3118                  &   0.47&   0.2113                   &   0.117            \\
                \bottomrule
            \end{tabular}
        \end{table}
        This table shows that for either trait, the epi test is not significant. However, trait ydose has a very significant in joint test, and One gene test for both gene A and gene B. Can we use it as an application of One gene test?

    \section{Debug}
        There is a weird problem I met when I debug the code for Epi test. Although now I directly use the similarity matrix $S_A$ and $S_B$ to estimate $\tau_A$ and $\tau_B$ through EM algorithm, which should get rid of the haplotype $H$ and the variance matrix $R$ for the haplotype effect, I still use the way you used in your one gene Trait-Similarity regression model to compute a fake $H$ and $R$ to facilitate the computation for $V^{-1}$. The $H_B$ and $R_B$ from $S_B$ can not fully satisfy the restriction that $S_B=H_BR_BH^T_B$, I think that is caused by the numerical calculation error. The weird part is, if I directly use $S_B$ to estimate $tau_B$ in the loop, $\tau_B$ can not converge, however, if I compute $S^*_B=H_BR_BH^T_B$ and replace $S_B$ by $S^*_B$ in the loop, everything is fine. I don't know whether it is big question since $S^*_B$ is quite close to $S_B$. But it is better to point out.        
        \begin{boxedminipage}{\textwidth}
        Q1: Does this problem only occur with GxG test or also for one G test conditional on another G effect? And this only occurs with this dataset or also occurred in simulation?

        A1: No, Both for the One Gene test and the simulation, everything is fine.
            
        Q2: Can you show me how you get $\tau_B$ directly? Wouldn't the dimension too big to calculate $V^{-1}$?
        
        A2: 
        
        Q3: To resolve this, you may want to do a small scale of simulation to make sure the type I error and estimate of tau B are ok when you use fake H and fake R. 
        
        A3: I will double check that.

        \end{boxedminipage}


\end{document} 