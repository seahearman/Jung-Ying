\documentclass{article}
\usepackage{booktabs}
\renewcommand{\baselinestretch}{1.1}
\usepackage{amsthm,amsmath, amssymb, amsfonts, amscd, xspace, pifont, color}
\usepackage{boxedminipage}
\usepackage{epsfig, psfrag}
\usepackage{multirow}
\usepackage[top=1in, bottom=1in, left=1in, right=1in]{geometry}
\usepackage{rotating}
\usepackage[round]{natbib}

\newtheorem{Theorem}{Theorem}[section]
\newtheorem{Lemma}[Theorem]{Lemma}

\newcommand{\FF}{\mathcal{F}}
\newcommand{\GG}{\mathcal{G}}
\newcommand{\LL}{\mathcal{L}}
\newcommand{\RR}{\mathbb{R}}
\newcommand{\Y}{\mathbf{Y}}
\newcommand{\X}{\mathbf{X}}
\newcommand{\Q}{\mathbf{Q}}
\newcommand{\HH}{\mathbf{H}}
\newcommand{\V}{\mathbf{V}}
\newcommand{\VV}{\mathbf{V}^{-1}}
\newcommand{\argmax}{\mathop{\arg\max}}
\newcommand{\argmin}{\mathop{\arg\min}}


\begin{document}

    \begin{center}
        \Large{The Derivation of EM alogrithm}
    \end{center}

    \fontsize{11pt}{\baselineskip}\selectfont
    The model is 
    \[
        Y=X\gamma+G_A+G_B+e
    \]
    
    Y: $N\times1$ matrix, records trait value.
    
    X: $N\times p$ matrix, records the covariant.
    
    $\gamma$: $p\times 1$, the effect of covariant.
    
    $G_A$: the gene effect for Gene A, treated as an random effect. $G_A\sim N(0, \tau_AS_A)$, $S_A$ is the similarity matrix which records the genetic similarity between individuals.

    $G_B$: the gene effect for Gene B, also treated as an random effect. $G_A$ and  $G_B$ are assumed to be independent.

    $e$: $N\times1$ matrix,  the error term. $e\sim N (0, \sigma I)$.

    Define $U=A^TY$ with the restriction that $A^TA=I_{N-p}$ and $AA^T=I-P_X$.

    It is easy to find out that : $E(U)=0$ and $Var(U)=A^TVA$ where $V=Var(Y)=\tau_AS_A+\tau_BS_B+\sigma I$.

    \[ \begin{split}
        Cov(U,G_A)=& Cov(A^TX\gamma+A^TG_A+A^TG_B+A^Te,G_A)\\
                  =& Cov(A^TG_A,G_A)\\
                  =& A^TCov(G_A,G_A)\\
                  =&\tau_AA^TS_A
    \end{split}\]

    In the same way, we have$Cov(U,G_B)=\tau_BA^TS_B$ and $Cov(G_A,G_B)=0$ since they are independent.

    Therefore, the joint distribution of $(U,G_A,G_B)^T$ is
    \[
        \begin{pmatrix} U\\G_A\\G_B \end{pmatrix} \sim MN \begin{pmatrix} \mu=\begin{pmatrix} 0\\0\\0 \end{pmatrix},\quad \Sigma=\begin{pmatrix} A^TVA&\tau_AA^TS_A&\tau_BA^TS_B\\ \tau_AS_AA&\tau_AS_A&0\\\tau_BS_BA&0&\tau_BS_B \end{pmatrix} \end{pmatrix}
    \]
    
    so we can have the following conditional mean and variance,
    \begin{enumerate}
        \item the conditional mean and variance for $U$ are
            \[\begin{split}
                E(U|G_A,G_B)=&A^T(G_A+G_B)\\
                Var(U|G_A,G_B) = & \sigma I_{N-p} 
            \end{split}\]
        \item the conditional mean and variance for $G_A$, since $Cov(G_A,G_B)=0$
            \[\begin{split}
                E(G_A|G_B,U)=&E(G_A|U)\\
                                           =&\tau_AS_AA(A^TVA)^{-1}A^TY\\
                Var(G_A|G_B,U)=&Var(G_A|U)\\
                                               =&\tau_AS_A-\tau_A^2S_AA(A^TVA)^{-1}A^TS_A
            \end{split}\]
            Simple algebra shows that $A(A^TVA)^{-1}A^T=P=V^{-1}-V^{-1}X(X^TV^{-1}X)^{-1}X^TV^{-1}$, so that
            \[\begin{split}
                E(G_A|G_B,U)=&\tau_AS_APY=g_A\\
                Var(G_A|G_B,U)=&\tau_AS_A-\tau_A^2S_APS_A=v_A
            \end{split}\]
        \item Similarly, the conditional mean and variance for $G_B$ are 
             \[\begin{split}
                E(G_B|G_A,U)=&\tau_BS_BPY=g_B\\
                Var(G_B|G_A,U)=&\tau_BS_B-\tau_B^2S_BPS_B=v_B
            \end{split}\]           
    \end{enumerate}
        
    Define $\theta=\{\sigma,\tau_A,\tau_B\}$, according to the EM algorithm, we need to first compute the $\log L(\theta^{(t)}|U,G_A,G_B)$,
    \[\begin{split}
        \log L(\theta^{(t)}|U,G_A,G_B)=&f(U,G_A,G_B|\theta^{(t)})\\
                                                                 =&\log f(U|G_A,G_B,\theta^{(t)})+\log f(G_A|\theta^{(t)})+\log f(G_B|\theta^{(t)})
    \end{split}\]
    
    Since $S_A$ and $S_B$ are singular, we have,
    \[
        f(G_A)=\dfrac{1}{((2\pi)^{rank(S_A)}|\tau_AS_A|_+)^{\frac{1}{2}}}\exp\Big (-\dfrac{1}{2}G_A^T(\tau_AS_A)^-G_A\Big)
    \]

    where $|\tau_AS_A|_+$ is the Pseudo-Determinant, and $(\tau_AS_A)^-)$ is the Generalized inverse.
    
    Define $rank(S_A)=q_A$, $rank(S_B)=q_B$, we have,
    \[\begin{split}
        f(G_A)=&\dfrac{1}{((2\pi)^{q_A}\tau_A^{q_A}|S_A|_+)^{\frac{1}{2}}}\exp\Big (-\dfrac{1}{2\tau_A}G_A^T(S_A)^-G_A\Big)\\
        f(G_B)=&\dfrac{1}{((2\pi)^{q_B}\tau_B^{q_B}|S_B|_+)^{\frac{1}{2}}}\exp \Big(-\dfrac{1}{2\tau_B}G_B^T(S_B)^-G_B\Big)
    \end{split}\]
    
    therefore,
    
    \[\begin{split}
       \log f(G_A)=&constant -\dfrac{q_A}{2}\log \tau_A-\dfrac{1}{2}\log(|S_A|_+)-\dfrac{1}{2\tau_A}G_A^TS_A^-G_A\\
       \log f(G_B)=&constant -\dfrac{q_B}{2}\log \tau_B-\dfrac{1}{2}\log(|S_B|_+)-\dfrac{1}{2\tau_B}G_B^TS_B^-G_B\\
       \log f(U|G_A,G_B) =& constant-\dfrac{N-p}{2}\log\sigma-\dfrac{1}{2\sigma}\Big[(Y-G_A-G_B)^T(I-P_X)(Y-G_A-G_B)   \Big]
    \end{split}\]
    
    EM algorithm treats $G_A$ and $G_B$ as missing values. So instead of estimating $G_A$ and $G_B$ and then plugging them into the $\log L$, EM algorithm calculate the expectation of $\log L$ given $U$ and $\theta^{(t-1)}$, then based on $E(\log L|U,\theta^{(t-1)})$, $\theta^{(t)}$ are calculated by taking partial derivative.
    
    \[
        E\Big(\log L\Big|U,\theta^{(t-1)}\Big)=E\Big(\log f(G_A)\Big|U,\theta^{(t-1)}\Big)+E\Big(\log f(G_B)\Big|U,\theta^{(t-1)}\Big)+E\Big(\log f(U|G_A,G_B)\Big|\theta^{(t-1)}\Big)
    \]
    
    It is easy to find out that to estimate $\tau_A^{(t)}$, we just need to consider $E\Big(\log f(G_A)\Big|U,\theta^{(t-1)}\Big)$, so let 
    \[\frac{\partial E\Big(\log f(G_A)\Big|U,\theta^{(t-1)}\Big)}{\partial\tau_A}=0\] 
    we have,
    \[
        -\dfrac{q_A}{2\tau_A}+\dfrac{1}{2\tau_A^2}E\Big(G_A^TS_A^-G_A\Big|U,\theta^{(t-1)}  \Big)=0
    \]
    
    \[
        E\Big(G_A^TS_A^-G_A\Big|U,\theta^{(t-1)} \Big)=\big(g_A^{(t-1)}\big)^TS_A^-g_A^{(t-1)}+tr(S_A^-v_A^{(t-1)})
    \]
    plugging the expression of $g_A^{(t-1)}$ and $v_A^{(t-1)}$, finally we have 
    \[
        \tau_A^{(t)}=\tau_A^{(t-1)}+\dfrac{[\tau_A^{t-1}]^2}{q_A}\Big[Y^TPS_APY-tr(S_AP) \Big]
    \]
    In the same way, we can estimate $\tau_B^{(t)}$ by 
    \[
        \tau_B^{(t)}=\tau_B^{(t-1)}+\dfrac{[\tau_B^{t-1}]^2}{q_B}\Big[Y^TPS_BPY-tr(S_BP) \Big]
    \]    
    To estimate $\sigma^{(t)}$, we just need to consider $E\Big(\log f(U|G_A,G_B)\Big|\theta^{(t-1)}\Big)$, so the final expression of $\sigma^{(t)}$ is 
    \[
        \sigma^{(t)}=\dfrac{1}{N-p}\Big\{\Big[(Y^*)^T(I-P_X)Y^*\Big] + tr\Big[(I-P_X)\Big(\tau_A^{(t-1)}S_A-(\tau_A^{(t-1)})^2S_APS_A+\tau_B^{(t-1)}S_B-(\tau_B^{(t-1)})^2S_BPS_B\Big)\Big]         \Big\}
    \]
    where $Y^*=Y-\tau_A^{(t-1)}S_APY-\tau_B^{(t-1)}S_BPY$
    
\end{document} 